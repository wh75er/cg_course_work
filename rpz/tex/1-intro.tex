\Introduction

Цвет - один из важнейших атрибутов визуальной информации. Во многих вещах мы полагаемся на цвета: будь то светофор, кино, фотографии, картины. Наличие каких-то конкретных оттенков задает нужное настроение и атмосферу. 

С каждым годом растет объем информации, которую нужно уметь качественно и быстро обрабатывать. Поэтому поисковые системы стараются расширить функционал своих продуктов, чтобы удовлетворить совершенно различные требования и желания пользователя. Одним из таких расширений является использование преобладающих цветов изображения. Данную технологю поисковые системы могут использовать в совершенно разных ключах. Помимо поиска по тегам(ключевым словам) поиск проводится по цветовому критерию. Например, мы хотим найти изображения по тегу 'футбол' и указываем доминирующий цвет - 'зеленый'. Таким образом, мы получаем множество картинок, которые с большой вероятностью содержат футбольное поле. Другой пример поиск  по тегу 'море' с преобладающим цветом - 'красный'. Здесь мы уже получим изображения, которые содержат море, и пребладающий красный оттенок(вероятнее всего это будет море и закат). 

Помимо поисковых систем, преобладающий цвет изображения может использоваться и в других продуктах. Системы, которые выводят визуальную информацию на экраны компьютера или телевизора могут автоматически генерировать подсветку вокруг экрана, которая будет соответствовать доминирующим цветам текущего отображаемого изображения. Такая технология позволяет увеличить эффект присутствия и позволяет снизить утомляемость глаз во время темного времени суток.

Оба этих рассмотренных примера работают с огромными объемами информации,что требует умения быстро обработать поступаемые данные.

Так же стоит выделить понятие преобладающий цвет, которое так часто использовалось выше. Цвет может быть преобладающим чисто математически, физически, а может быть преобладающим с точки зрения человека. Когда мы смотрим на изображение, в котором 70\% черного цвета, а остальные 30\% - яркий выразительный оранжевый оттенок, мы можем выделить как раз таки эти два разных преобладающих цвета. В первом случае мы возьмем преобладающий цвет как нечто физическое и скажем, что в нашем изображении доминирующий цвет - черный, потому что он занимает большую часть картники. А во втором случае рассмотрим преобладающий цвет с точки зрения человека, где скажем, что оранжевый преобладает, потому что наш глаз в первую очередь обратит внимание на более яркую, выразительную точку, чем темную и тусклую.
