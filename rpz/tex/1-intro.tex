\Introduction

В 20 веке был принят стандарт CIE, который охарактеризовывал то, как должна отображаться картинка на восьмибитных мониторах. Технологии того времени позволяли передавать только сильно ограниченный диапазон цветовой информации, поэтому картинка, предоставляемая монитором, была далека от более сочных и ярких цветов, которые человек видит в повседневной жизни. Такое явление может быть охарактеризованно следующими терминами: LDR(low dynamic range) или SDR(standart dynamic range) image -- маленький динамичиский диапазон или стандартный динамичиский диапазон изображения. Мониторы LDR и SDR не могли передавать широкий спектр цветов. К тому же не все сенсоры цифровых камер могут позволить себе захватить широкий диапазон оттенков. Из-за этого возникает проблема засвеченных и затемненных участков на фотографии или видео.

С развитием технологий, стали появляться так называемые HDR мониторы, которые позволяют передавать цветовую информацию с глубиной в 10 битов. По причине того, что LDR и SDR мониторы обладают маленькой глубиной цвета, они не в состоянии корректно отображать картинку с широким диапазоном цветов.

Из-за того, что не все сенсоры могут запечатлить широкий спектр цветов, были придуманы специальные алгоритмы и методы получения HDR изображений с обычных и самых распространенных сенсоров(цифровые камеры, мобильные телефоны, планшеты, веб-камеры и т.п.). Полученные HDR снимки или видео с камер, сенсор которых рассчитан на LDR и SDR изображения, могут быть отображены на HDR мониторах.

Получение HDR кадра - нетривиальная задача, которая делится на несколько этапов: получения серии снимков с разной длиной экспозиции, выравниване кадров по отношению друг к другу, реконструкция и удаление движущихся объектов полученных изображений, объединение изображений(merging), проведение цветовой коррекции изображения.
