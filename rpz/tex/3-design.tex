\chapter{ Констукторский раздел}
\label{cha:design}

Рассмотрим задачу выделения преобладающих цветов в режиме реального времени в большом потоке данных. 

\section{ Идея}
При поступлении большого потока данных, требуется сразу выделять доминтные цвета. Каждый раздел, который характеризует преобладающий цвет, должен иметь свой вес -- частоту возникновения в изображении. Далее должно происходить квантования цветов. Здесь очень многое зависит от самого алгоритма, ведь он должен работать не только быстро, но и более менее правильно. После завершения остается выдать самый весомый преобладающий цвет и продолжить работу со следующим изображением в очереди.

\section{ Алгоритм}
Изображение с большим разрешеним имеет много пикселей. С точки зрения цветов большое количество пикселей может быть избыточно, поэтому для их уменьшение изображения следует уменьшить. Это позволит алгоритму квантования обрабатывать более меньшие объемы. На этапе анализа были приведены нескольколько алгоритмов и выбор был сделан в сторону самого быстрого алгоритма -- LBA. Данный алгоритм очень быстр и эффективен с математической точки зрения. Позволяет выделить только самые основные цвета, которые не пересекаются между собой, что связано с методом слияния кластеров. 
