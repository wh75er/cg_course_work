\chapter{ Аналитический раздел}
\label{cha:analysis}

\textit{SDR(Standart Dynamic Range) изображение} -- изображение, пиксели которого содержат цвета и яркость, соответствующую глубине монитора.

\textit{LDR(Low Dynamic Range) изображение} -- изображение, пиксели которого хранят ограниченный диапазон цветов и яркости, предназначенное для отображении на старых мониторах.

\textit{HDR(High Dynamic Range) изображение} -- изображение, пиксели которого содержат более широкие значения цвета и яркости в сравнении с изображениями стандартного диапазона(SDR).

Получить HDR изображение можно получить несколькими способами: 
\begin{itemize}
    \item c помощью объединения снимков с разной длинной экспозиции,
    \item с помощью камеры, сенсор который позволяет захватить широкий объем данных,
    \item при помощи перевода LDR изображения в HDR специальными алгоритмами
\end{itemize}

Первый метод является более распространенным, так как устройства, которые больше всего распространены в повседневной жизни(телефоны, планшеты, веб-камеры) не обладают достаточно мощными сенсорами, для того, чтобы захватить широкий диапазон цветов. Последний метод не получил распространения, потому что задача перевода LDR или SDR изображения в HDR возможна только при помощи преминениями алгоритмов реконструкций, завязанных нейронных сетях, появивщихся достаточно недавно.

Задача получения HDR изображения не является тривиальной и делится на несколько этапов:
%\begin{itemize}
%\end{itemize}
